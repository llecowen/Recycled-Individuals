\documentclass[]{article}
\usepackage{lmodern}
\usepackage{amssymb,amsmath}
\usepackage{ifxetex,ifluatex}
\usepackage{fixltx2e} % provides \textsubscript
\ifnum 0\ifxetex 1\fi\ifluatex 1\fi=0 % if pdftex
  \usepackage[T1]{fontenc}
  \usepackage[utf8]{inputenc}
\else % if luatex or xelatex
  \ifxetex
    \usepackage{mathspec}
  \else
    \usepackage{fontspec}
  \fi
  \defaultfontfeatures{Ligatures=TeX,Scale=MatchLowercase}
  \newcommand{\euro}{€}
\fi
% use upquote if available, for straight quotes in verbatim environments
\IfFileExists{upquote.sty}{\usepackage{upquote}}{}
% use microtype if available
\IfFileExists{microtype.sty}{%
\usepackage{microtype}
\UseMicrotypeSet[protrusion]{basicmath} % disable protrusion for tt fonts
}{}
\usepackage[margin=1in]{geometry}
\usepackage{hyperref}
\PassOptionsToPackage{usenames,dvipsnames}{color} % color is loaded by hyperref
\hypersetup{unicode=true,
            pdftitle={Web-based Supplementary Materials for The Effect of Recycled Individuals in the Jolly-Seber Tag Loss Model},
            pdfborder={0 0 0},
            breaklinks=true}
\urlstyle{same}  % don't use monospace font for urls
\usepackage{graphicx,grffile}
\makeatletter
\def\maxwidth{\ifdim\Gin@nat@width>\linewidth\linewidth\else\Gin@nat@width\fi}
\def\maxheight{\ifdim\Gin@nat@height>\textheight\textheight\else\Gin@nat@height\fi}
\makeatother
% Scale images if necessary, so that they will not overflow the page
% margins by default, and it is still possible to overwrite the defaults
% using explicit options in \includegraphics[width, height, ...]{}
\setkeys{Gin}{width=\maxwidth,height=\maxheight,keepaspectratio}
\setlength{\parindent}{0pt}
\setlength{\parskip}{6pt plus 2pt minus 1pt}
\setlength{\emergencystretch}{3em}  % prevent overfull lines
\providecommand{\tightlist}{%
  \setlength{\itemsep}{0pt}\setlength{\parskip}{0pt}}
\setcounter{secnumdepth}{0}

%%% Use protect on footnotes to avoid problems with footnotes in titles
\let\rmarkdownfootnote\footnote%
\def\footnote{\protect\rmarkdownfootnote}

%%% Change title format to be more compact
\usepackage{titling}

% Create subtitle command for use in maketitle
\newcommand{\subtitle}[1]{
  \posttitle{
    \begin{center}\large#1\end{center}
    }
}

\setlength{\droptitle}{-2em}
  \title{Web-based Supplementary Materials for The Effect of Recycled Individuals
in the Jolly-Seber Tag Loss Model}
  \pretitle{\vspace{\droptitle}\centering\huge}
  \posttitle{\par}
  \author{}
  \preauthor{}\postauthor{}
  \date{}
  \predate{}\postdate{}



% Redefines (sub)paragraphs to behave more like sections
\ifx\paragraph\undefined\else
\let\oldparagraph\paragraph
\renewcommand{\paragraph}[1]{\oldparagraph{#1}\mbox{}}
\fi
\ifx\subparagraph\undefined\else
\let\oldsubparagraph\subparagraph
\renewcommand{\subparagraph}[1]{\oldsubparagraph{#1}\mbox{}}
\fi

\begin{document}
\maketitle

\begin{quote}
\begin{center} {\large Emily Malcolm-White and Laura L.E. Cowen} \end{center}
\end{quote}

\section{Web Appendix A: The Jolly-Seber Tag Loss
Model}\label{web-appendix-a-the-jolly-seber-tag-loss-model}

\subsubsection{Assumptions}\label{assumptions}

Assumptions of the JSTL model with constant survival, capture, and tag
retention probabilities and time-varying entry probabilities are as
follows:

\begin{itemize}
\tightlist
\item
  The effect of recycled individuals is negligible
\item
  All individuals (marked and unmarked) are equally catchable, and that
  capture probabilities for all individuals are the same for all
  individuals at all sample time
\item
  All individuals (marked and unmarked) have equal survival
  probabilities between all sample times
\item
  All individuals have equal entry (birth or immigration) probabilities,
  but entry probabilities can vary between sample times
\item
  All marked individuals have equal tag retention probabilities between
  all sample times
\item
  For double-tagged individuals, tag loss is independent between tags
\item
  There is independence across all individuals
\item
  The sampling period is relatively short compared to the interval
  between sampling times
\end{itemize}

\subsubsection{Notation}\label{notation}

The following is a description of statistics and parameters used in the
JSTL model where survival, capture, and tag retention probabilities are
assumed to be constant across time.

\emph{Statistics:}\\
\[
  \begin{array}{ll}
      k= & \text{number of sample times} \\
      m= & \text{number of unique tag histories} \\
      i= & \text{index for a tag history, $i=0,1,2,...,m.$ $i=0$ refers to the tag history of individuals never captured}\\
      nt_j= & \text{the number of tags on individuals at sample time \textit{j, j=1,2,..,k}.} \\
      n_j= & \text{total individuals captured at time \textit{j}} \\
      n_{\text{obs}}= & \text{the total number of individuals captured with no tags and treated as new individuals; when no}\\
      & \text{recycling is present, $n_\text{obs}$ is the number of unique individuals observed throughout the study}\\
      f_i= & \text{the first sample time where the individual with tag history \textit{i} was captured}\\
      l_i= & \text{the last sample time where the individual with tag history \textit{i} was observed} \\
      q_{id}= & \text{first sample time where tag $d$ was known missing for the individual with capture history $i$} \\
      l_{id}= & \text{the last sample time where tag \textit{d} is present on the individual with tag history \textit{i}, } d=1,2 \text{ for a}\\
      & \text{double-tagging experiment} \\
      \omega_i*= & \text{capture history vector } \omega_i*=(\omega_{i1}*,\omega_{i2}*,...,\omega_{ik}*), \textit{i=0,1,...,m} \text{, where}\\
       & \omega_{ij}* = \left\{\begin{array}{ll}
                        1 & \text{if the individual with tag history \textit{i} captured for the first time at sample time \textit{j}.} \\
                        1 & \text{if the individual with tag history \textit{i} captured at sample time }f_i<j \text{ with at least }\\
                        & \text{one tag present.}\\
                        0 & \text{if the individual with tag history \textit{i} not captured at sample time \textit{j}.} \\
                      \end{array}\right. \\
      \omega_i= & \text{tag history vector } \omega_i=(\omega_{i,1,1},\omega_{i,1,2},\omega_{i,2,1,},\omega_{i,2,2},...,\omega_{i,k,1},\omega_{i,k,2}), \textit{i=0,1,...,m}\text{, where} \\ 
       & \omega_{ijd}* = \left\{\begin{array}{ll}
                          1 & \text{if the individual with tag history \textit{i} captured and tag \textit{d} was present at sample time \textit{j}.} \\
                          0 & \text{if the individual with tag history \textit{i} captured and tag \textit{d} was not present at sample time \textit{j}.}\\
                          0 & \text{if the individual with tag history \textit{i} not captured at sample time \textit{j}.} \\
                      \end{array}\right. \\
      n_{\omega_i=} & \text{number of } \omega_i \text{ tag histories} \\
  \end{array}
\]

\[
  \begin{array}{ll}
      p= & \text{the probability that an individual is recaptured at a sample time given that the individual was alive}\\
      & \text{at the previous sample time} \\
      \phi=  & \text{the probability that an individual survives and remain in the population between a sample time}\\
      & \text{and the next sample time }\\
      b_j= & \text{the probability that an individual enters the system between sample times $j$ and $j+1$. $j=0,1,...,k-1$.}\\
       & \text{$b_0$ is the expected fraction of individuals alive just prior to the first sample time.}\\
      \upsilon= & \text{the probability that an individual captured will be lost on capture}\\
      T_d= & \text{the probability that an individual is marked with $d$ tags. Note that the probability of marking with a }\\
      & \text{single tag is one minus the probability of marking with a double tag: $T_1=1-T_2$}\\
      \lambda= & \text{the probability that an individual captured will  retain its tag between time periods given that}\\
      & \text{it remains alive}\\
      N= & \text{super-population size, the total number of individuals ever present in population and available for}\\
      & \text{capture during the study}\\
  \end{array}
\]

\emph{Functions of Parameters:}\\
\[
  \begin{array}{ll}
    b_j*= & \text{the expected fraction of the population remaining to enter the population that enters between}\\
    & \text{sample times $j$ and $j+1$, $j=0,1,...,k-1$.}\\
       & b_{j}* = \left\{\begin{array}{ll}
                          b_0 & \text{if $j=0$} \\
                          b_j / \sum_{u=j}^{k-1} b_u & \text{if $j=1,...,k-1$}\\
                          1 & \text{if $j=k-1$} \\
                      \end{array}\right. \\
    B_j= & \text{net births; the number of individuals who enter the population after sample time $j$ and survive}\\
    & \text{to sample time $j+1$; $j=0,1,...,k-1$. $B_0$ is the number of indiividuals alive just before the}\\
    & \text{first sample time. Note that $E(B_j|N)=Nb_j$. }\\
    \chi_{(f_i,l_i,nt)}= & \text{the probability that the individual with capture history $i$ is first seen at $f_i$ and not seen after}\\
    & \text{sample time $l_i$, with $nt$ tags. This is a recursive function of $\phi$, $p$, and $\lambda$. If $f_i=0$, this}\\
    & \text{indicates individuals not yet captured but alive at time $l_i$.}\\
    & \text{For individuals not yet captured:}\\
       & \chi_{(0,j,0)} = \left\{\begin{array}{ll}
                          1-\phi+\phi(1-p)\chi_{(0,j+1,0)} & \text{if $j<k$} \\
                          1 & \text{if $j=k$}\\
                      \end{array}\right. \\
    & \text{For single tagged individuals:}\\
       & \chi_{(f_i,j,1)} = \left\{\begin{array}{ll}
                          1-\phi+\phi(1-p)\lambda \chi_{(f_i,j+1,1)}+\phi(1-\lambda) & \text{if $j<k$} \\
                          1 & \text{if $j=k$}\\
                      \end{array}\right. \\
    & \text{For double tagged individuals:}\\
       & \chi_{(f_i,j,2)} = \left\{\begin{array}{ll}
                          1-\phi+\phi(1-p)\lambda^2 \chi_{(f_i,j+1,2)}+\phi(1-\lambda)^2+2\phi(1-p)\lambda(1-\lambda)\chi_{(f_i,j+1,1)} & \text{if $j<k$} \\
                          1 & \text{if $j=k$}\\
                      \end{array}\right. \\
    \psi_j= & \text{probability that an individual enters the population, is still alive and is not seen before time $j$; }\\
    &j=1,2,...,k\\ 
    N_j= & \text{population size at time $j$. E$(N_1|N)=B_0$, E$(N_{j+1}|N)=(N_j-N_j p \upsilon)\phi +B_j$, which is the }\\
    & \text{number of individuals that survive from time $j$ minus the number lost on capture plus the }\\
    & \text{number of births.}\\
  \end{array}
\]

\subsubsection{Likelihood and Estimation}\label{likelihood-and-estimation}

The likelihood of the JSTL model can be divided into three parts: the
probability of observing \(n_{\text{obs}}\), the number of tag
histories, given the super-population size (\(L_1^A\)), the probability
of observing recaptures given the number of tag histories (\(L_1^B\)),
and the probability of observing the number of individuals lost on
capture (\(L_3\)).

The probability of observing \(n_\text{obs}\) capture histories is given
by a binomial distribution conditional on the super-population size. \[
L_1^A=[n_{\text{obs}}|N] \sim \text{Binomial} (N, 1-P_0) \text{, where}
\] \[
P_0\text{ is the probability of never being seen, given by } P_0=\sum_{j=0}^{k-1} b_j (1-p) \chi_{(0,j+1,0)}. 
\]

The probability of observing each unique tag history \(\omega_i\) is
modeled by a multinomial conditional upon being observed at least once.
\[
L_1^B=[n_{\omega_i}|n_{obs}] \sim \text{Multinomial} (n_{obs}, \pi_{\omega_i}) \text{, where} 
\] \[
\pi_{\omega_i}= P_{\omega_i}/(1-P_0) \text{ and}
\] \[
P(\omega_i)=\psi_{f_i} T_d \left\{ \prod_{j=f_i}^{l_i} p^{w*_{ij}} (1-p)^{(1-w*_{ij})} \right\} \left\{ \phi^{l_i-1-f_i} \right\} \times \prod_{d=1}^{2} \left\{ \left( \lambda^{l_{id}-1-f_i} \right) \left(1- \lambda^{q_{id}-1-l_{id}} \right)^{I(l_{id} \neq l_i)} \right\} \times \chi_{f_i, l_i, nt_{l_i}}
\]

The third component \(L_3\) models the number of losses on capture as a
binomial. In this study we assume there is no possibility of loss on
capture, this third component of the likelihood simplifies to 1. 


The complete likelihood for the Jolly-Seber tag loss model assuming no
possibility of loss of capture and assuming homogeneous survival,
capture, and tag retention probabilities is given below.

\[
L=\binom{N}{n_\text{obs}} \left\{ \sum_{j=0}^{k-1} b_j(1-p)\chi_{(0,j+1,0)} \right\} ^{(N-n_\text{obs})} \times \left\{ 1- \sum_{j=0}^{k-1} b_j(1-p)\chi_{(0,j+1,0)} \right\} ^{n_\text{obs}} \times
\] \[
\binom{n_\text{obs}}{n_{\omega_1},n_{\omega_2},...,n_{\omega_m}} \prod_{i=1}^{m} \Bigg[ \psi_{f_i} T_d \left\{ \prod_{j=f_i}^{l_i} p^{w*_{ij}} (1-p)^{(1-w*_{ij})} \right\} \left\{ \prod_{j=f_i}^{l_i-1} \phi \right\} \times 
\] \[
\prod_{d=1}^{2} \left\{ \left( \prod_{j=f_i}^{l_{id}-1} \lambda \right) \left(1-\prod_{j=l_{id}}^{q_{id}-1} \lambda \right)^{I(l_{id} \neq l_i)} \right\} \times \chi_{f_i, l_i, nt_{l_i}} \Bigg]^{n_{\omega_i}} \times 
\] \[
\left\{ 1- \sum_{j=0}^{k-1} b_j(1-p)\chi_{(0,j+1,0)} \right\} ^{-n_\text{obs}}
\]

\newpage

\subsection{Recycled Individuals}\label{recycled-individuals}

\textsc{Web Table 1:} Examples of the fraction of recycled individuals
(number of recycled individuals captured / total number of individuals
captured) at each sample occasion for a 10 occasion experiment with
super-population size \(N=1000\) with 100\% double-tagging.

\begin{table}[ht]
\begin{tabular}{rrrrrrrrrrr}
  \hline
& \multicolumn{10}{c}{Sampling Times}\\
 & 1 & 2 & 3 & 4 & 5 & 6 & 7 & 8 & 9 & 10 \\ 
  \hline
  $\phi=0.9, p=0.9, \lambda=0.2$ & 0.00 & 0.30 & 0.36 & 0.47 & 0.51 & 0.53 & 0.54 & 0.55 & 0.56 & 0.57 \\ 
  $\phi=0.9, p=0.9, \lambda=0.5$ & 0.00 & 0.12 & 0.12 & 0.20 & 0.29 & 0.26 & 0.29 & 0.30 & 0.31 & 0.31 \\ 
  $\phi=0.9, p=0.9, \lambda=0.9$ & 0.00 & 0.01 & 0.02 & 0.02 & 0.01 & 0.01 & 0.03 & 0.04 & 0.04 & 0.03 \\ \hline
  $\phi=0.9, p=0.5, \lambda=0.2$ & 0.00 & 0.14 & 0.28 & 0.38 & 0.43 & 0.48 & 0.54 & 0.54 & 0.55 & 0.56 \\ 
  $\phi=0.9, p=0.5, \lambda=0.5$ & 0.00 & 0.02 & 0.12 & 0.14 & 0.27 & 0.28 & 0.28 & 0.33 & 0.29 & 0.38 \\ 
  $\phi=0.9, p=0.5, \lambda=0.9$ & 0.00 & 0.00 & 0.00 & 0.00 & 0.01 & 0.02 & 0.02 & 0.04 & 0.06 & 0.06 \\ \hline
  $\phi=0.9, p=0.2, \lambda=0.2$ & 0.00 & 0.05 & 0.15 & 0.35 & 0.18 & 0.28 & 0.30 & 0.33 & 0.40 & 0.37 \\ 
  $\phi=0.9, p=0.2, \lambda=0.5$ & 0.00 & 0.00 & 0.06 & 0.09 & 0.14 & 0.13 & 0.20 & 0.21 & 0.30 & 0.26 \\ 
  $\phi=0.9, p=0.2, \lambda=0.9$ & 0.00 & 0.00 & 0.00 & 0.00 & 0.00 & 0.01 & 0.02 & 0.02 & 0.04 & 0.03 \\ \hline
  $\phi=0.5, p=0.9, \lambda=0.2$ & 0.00 & 0.18 & 0.23 & 0.27 & 0.34 & 0.36 & 0.32 & 0.32 & 0.32 & 0.26 \\ 
  $\phi=0.5, p=0.9, \lambda=0.5$ & 0.00 & 0.10 & 0.10 & 0.15 & 0.10 & 0.15 & 0.18 & 0.13 & 0.19 & 0.12 \\ 
  $\phi=0.5, p=0.9, \lambda=0.9$ & 0.00 & 0.00 & 0.00 & 0.01 & 0.01 & 0.00 & 0.01 & 0.01 & 0.01 & 0.01 \\ \hline
  $\phi=0.2, p=0.9, \lambda=0.2$ & 0.00 & 0.10 & 0.14 & 0.07 & 0.15 & 0.10 & 0.11 & 0.13 & 0.13 & 0.12 \\ 
  $\phi=0.2, p=0.9, \lambda=0.5$ & 0.00 & 0.02 & 0.06 & 0.05 & 0.05 & 0.11 & 0.04 & 0.06 & 0.08 & 0.09 \\ 
  $\phi=0.2, p=0.9, \lambda=0.9$ & 0.00 & 0.01 & 0.00 & 0.01 & 0.00 & 0.00 & 0.00 & 0.00 & 0.00 & 0.00 \\ \hline
    $\phi=0.5, p=0.5, \lambda=0.5$ & 0.00 & 0.01 & 0.07 & 0.06 & 0.12 & 0.13 & 0.14 & 0.16 & 0.13 & 0.14 \\ \hline
\end{tabular}
\end{table}

\textsc{Web Table 2:} Examples of the fraction of recycled individuals
(number of recycled individuals captured / total number of individuals
captured) at each sample occasion for a 7 occasion experiment with
super-population size \(N=1000\) with 100\% double-tagging.

\begin{table}[ht]
\begin{tabular}{rrrrrrrr}
  \hline
  & \multicolumn{7}{c}{Sampling Times}\\
 & 1 & 2 & 3 & 4 & 5 & 6 & 7 \\ 
  \hline
  $\phi=0.9, p=0.9, \lambda=0.2$ & 0.00 & 0.32 & 0.40 & 0.49 & 0.50 & 0.53 & 0.55 \\ 
  $\phi=0.9, p=0.9, \lambda=0.5$ & 0.00 & 0.10 & 0.18 & 0.21 & 0.22 & 0.29 & 0.29 \\ 
  $\phi=0.9, p=0.9, \lambda=0.9$ & 0.00 & 0.01 & 0.01 & 0.02 & 0.02 & 0.02 & 0.02 \\ \hline
  $\phi=0.9, p=0.5, \lambda=0.2$ & 0.00 & 0.12 & 0.30 & 0.35 & 0.42 & 0.47 & 0.50 \\ 
  $\phi=0.9, p=0.5, \lambda=0.5$ & 0.00 & 0.06 & 0.15 & 0.22 & 0.26 & 0.24 & 0.27 \\ 
  $\phi=0.9, p=0.5, \lambda=0.9$ & 0.00 & 0.01 & 0.00 & 0.02 & 0.02 & 0.03 & 0.04 \\ \hline
  $\phi=0.9, p=0.2, \lambda=0.2$ & 0.00 & 0.05 & 0.12 & 0.15 & 0.22 & 0.21 & 0.30 \\ 
  $\phi=0.9, p=0.2, \lambda=0.5$ & 0.00 & 0.00 & 0.06 & 0.10 & 0.15 & 0.20 & 0.20 \\ 
  $\phi=0.9, p=0.2, \lambda=0.9$ & 0.00 & 0.00 & 0.00 & 0.00 & 0.00 & 0.01 & 0.01 \\ \hline
  $\phi=0.5, p=0.9, \lambda=0.2$ & 0.00 & 0.19 & 0.30 & 0.32 & 0.30 & 0.34 & 0.38 \\ 
  $\phi=0.5, p=0.9, \lambda=0.5$ & 0.00 & 0.09 & 0.12 & 0.13 & 0.15 & 0.11 & 0.14 \\ 
  $\phi=0.5, p=0.9, \lambda=0.9$ & 0.00 & 0.00 & 0.00 & 0.02 & 0.00 & 0.00 & 0.01 \\ \hline
  $\phi=0.2, p=0.9, \lambda=0.2$ & 0.00 & 0.10 & 0.11 & 0.12 & 0.13 & 0.16 & 0.08 \\ 
  $\phi=0.2, p=0.9, \lambda=0.5$ & 0.00 & 0.04 & 0.05 & 0.07 & 0.07 & 0.03 & 0.02 \\ 
  $\phi=0.2, p=0.9, \lambda=0.2$ & 0.00 & 0.00 & 0.00 & 0.00 & 0.00 & 0.00 & 0.00 \\ \hline
    $\phi=0.5, p=0.5, \lambda=0.5$ & 0.00 & 0.05 & 0.04 & 0.05 & 0.09 & 0.11 & 0.12 \\ \hline
\end{tabular}
\end{table}

\clearpage

\textsc{Web Table 3:} Examples of the fraction of recycled individuals
(number of recycled individuals captured / total number of individuals
captured) at each sample occasion for a 5 occasion experiment with
super-population size \(N=1000\) with 100\% double-tagging.

\begin{table}[ht]
\begin{tabular}{rrrrrr}
  \hline
  & \multicolumn{5}{c}{Sampling Times}\\
 & 1 & 2 & 3 & 4 & 5 \\ 
  \hline
 $\phi=0.9, p=0.9, \lambda=0.2$  & 0.00 & 0.29 & 0.34 & 0.45 & 0.52 \\ 
  $\phi=0.9, p=0.9, \lambda=0.5$  & 0.00 & 0.08 & 0.22 & 0.23 & 0.28 \\ 
  $\phi=0.9, p=0.9, \lambda=0.9$  & 0.00 & 0.01 & 0.01 & 0.02 & 0.02 \\ \hline 
  $\phi=0.9, p=0.5, \lambda=0.2$  & 0.00 & 0.14 & 0.27 & 0.33 & 0.38 \\ 
  $\phi=0.9, p=0.5, \lambda=0.5$  & 0.00 & 0.05 & 0.13 & 0.18 & 0.25 \\ 
  $\phi=0.9, p=0.5, \lambda=0.9$  & 0.00 & 0.00 & 0.01 & 0.01 & 0.02 \\ \hline
  $\phi=0.9, p=0.2, \lambda=0.2$  & 0.00 & 0.06 & 0.15 & 0.16 & 0.29 \\ 
  $\phi=0.9, p=0.2, \lambda=0.5$ & 0.00 & 0.03 & 0.06 & 0.08 & 0.12 \\ 
  $\phi=0.9, p=0.2, \lambda=0.9$ & 0.00 & 0.00 & 0.00 & 0.01 & 0.01 \\ \hline
  $\phi=0.5, p=0.9, \lambda=0.2$ & 0.00 & 0.14 & 0.25 & 0.30 & 0.33 \\ 
  $\phi=0.5, p=0.9, \lambda=0.5$ & 0.00 & 0.05 & 0.13 & 0.10 & 0.15 \\ 
  $\phi=0.5, p=0.9, \lambda=0.9$ & 0.00 & 0.00 & 0.01 & 0.01 & 0.00 \\ \hline
  $\phi=0.2, p=0.9, \lambda=0.2$ & 0.00 & 0.11 & 0.10 & 0.13 & 0.12 \\ 
  $\phi=0.2, p=0.9, \lambda=0.5$  & 0.00 & 0.02 & 0.03 & 0.05 & 0.04 \\ 
  $\phi=0.2, p=0.9, \lambda=0.9$  & 0.00 & 0.00 & 0.00 & 0.00 & 0.00 \\ \hline
    $\phi=0.5, p=0.5, \lambda=0.5$ & 0.00 & 0.03 & 0.09 & 0.10 & 0.12 \\ \hline
\end{tabular}
\end{table}

\clearpage

\subsection{Survival Estimates}\label{survival-estimates}

\includegraphics{Appendix_files/figure-latex/1_survival_GJSTL1-1.pdf}

\textsc{Web Figure 1:} Boxplots of survival estimates (\(\hat{\phi}\))
of 100 simulated datasets analyzed with and without the effect of
recycled individuals for population size \(1000\) with \(T_2=1\) with 10
time periods for varying survival (\(\phi=0.2,0.5,0.9\)), capture
(\(p=0.2,0.5,0.9\)), and tag retention (\(\lambda=0.2,0.5,0.9\))
probabilities. The black line indicates the true value of \(\phi\) used
to simulate the data for each model.

\includegraphics{Appendix_files/figure-latex/2_survival_GJSTL2-1.pdf}

\textsc{Web Figure 2:} Boxplots of survival estimates (\(\hat{\phi}\))
of 100 simulated datasets analyzed with and without the effect of
recycled individuals for population size \(100000\) with \(T_2=1\) with
10 time periods for varying survival (\(\phi=0.2,0.5,0.9\)), capture
(\(p=0.2,0.5,0.9\)), and tag retention (\(\lambda=0.2,0.5,0.9\))
probabilities. The black line indicates the true value of \(\phi\) used
to simulate the data for each model.

\newpage

\includegraphics{Appendix_files/figure-latex/3_survival_GJSTL4-1.pdf}

\textsc{Web Figure 3:} Boxplots of survival estimates (\(\hat{\phi}\))
of 100 simulated datasets analyzed with and without the effect of
recycled individuals for population size \(1000\) with \(T_2=0.5\) with
10 time periods for varying survival (\(\phi=0.2,0.5,0.9\)), capture
(\(p=0.2,0.5,0.9\)), and tag retention (\(\lambda=0.2,0.5,0.9\))
probabilities. The black line indicates the true value of \(\phi\) used
to simulate the data for each model.

\includegraphics{Appendix_files/figure-latex/4_survival_GJSTL3-1.pdf}

\textsc{Web Figure 4:} Boxplots of survival estimates (\(\hat{\phi}\))
of 100 simulated datasets analyzed with and without the effect of
recycled individuals for population size \(100000\) with \(T_2=0.5\)
with 10 time periods for varying survival (\(\phi=0.2,0.5,0.9\)),
capture (\(p=0.2,0.5,0.9\)), and tag retention (\(\lambda=0.2,0.5,0.9\))
probabilities. The black line indicates the true value of \(\phi\) used
to simulate the data for each model.

\newpage

\includegraphics{Appendix_files/figure-latex/5_survival_GJSTL5-1.pdf}

\textsc{Web Figure 5:} Boxplots of survival estimates (\(\hat{\phi}\))
of 100 simulated datasets analyzed with and without the effect of
recycled individuals for population size \(1000\) with \(T_2=1\) with 5
time periods for varying survival (\(\phi=0.2,0.5,0.9\)), capture
(\(p=0.2,0.5,0.9\)), and tag retention (\(\lambda=0.2,0.5,0.9\))
probabilities. The black line indicates the true value of \(\phi\) used
to simulate the data for each model.

\includegraphics{Appendix_files/figure-latex/6_survival_GJSTL6-1.pdf}

\textsc{Web Figure 6:} Boxplots of survival estimates (\(\hat{\phi}\))
of 100 simulated datasets analyzed with and without the effect of
recycled individuals for population size \(1000\) with \(T_2=1\) with 7
time periods for varying survival (\(\phi=0.2,0.5,0.9\)), capture
(\(p=0.2,0.5,0.9\)), and tag retention (\(\lambda=0.2,0.5,0.9\))
probabilities. The black line indicates the true value of \(\phi\) used
to simulate the data for each model.

\newpage

\subsection{Capture Estimates}\label{capture-estimates}

\includegraphics{Appendix_files/figure-latex/7_capture_GJSTL1-1.pdf}

\textsc{Web Figure 7:} Boxplots of capture estimates (\(\hat{p}\)) of
100 simulated datasets analyzed with and without the effect of recycled
individuals for population size \(1000\) with \(T_2=1\) with 10 time
periods for varying survival (\(\phi=0.2,0.5,0.9\)), capture
(\(p=0.2,0.5,0.9\)), and tag retention (\(\lambda=0.2,0.5,0.9\))
probabilities. The black line indicates the true value of \(p\) used to
simulate the data for each model.

\includegraphics{Appendix_files/figure-latex/8_capture_GJSTL2-1.pdf}

\textsc{Web Figure 8:} Boxplots of capture estimates (\(\hat{p}\)) of
100 simulated datasets analyzed with and without the effect of recycled
individuals for population size \(100000\) with \(T_2=1\) with 10 time
periods for varying survival (\(\phi=0.2,0.5,0.9\)), capture
(\(p=0.2,0.5,0.9\)), and tag retention (\(\lambda=0.2,0.5,0.9\))
probabilities. The black line indicates the true value of \(p\) used to
simulate the data for each model.

\newpage

\includegraphics{Appendix_files/figure-latex/9_capture_GJSTL4-1.pdf}

\textsc{Web Figure 9:} Boxplots of capture estimates (\(\hat{p}\)) of
100 simulated datasets analyzed with and without the effect of recycled
individuals for population size \(1000\) with \(T_2=0.5\) with 10 time
periods for varying survival (\(\phi=0.2,0.5,0.9\)), capture
(\(p=0.2,0.5,0.9\)), and tag retention (\(\lambda=0.2,0.5,0.9\))
probabilities. The black line indicates the true value of \(p\) used to
simulate the data for each model.

\includegraphics{Appendix_files/figure-latex/10_capture_GJSTL3-1.pdf}

\textsc{Web Figure 10:} Boxplots of capture estimates (\(\hat{p}\)) of
100 simulated datasets analyzed with and without the effect of recycled
individuals for population size \(100000\) with \(T_2=0.5\) with 10 time
periods for varying survival (\(\phi=0.2,0.5,0.9\)), capture
(\(p=0.2,0.5,0.9\)), and tag retention (\(\lambda=0.2,0.5,0.9\))
probabilities. The black line indicates the true value of \(p\) used to
simulate the data for each model.

\newpage

\includegraphics{Appendix_files/figure-latex/11_capture_GJSTL5-1.pdf}

\textsc{Web Figure 11:} Boxplots of capture estimates (\(\hat{p}\)) of
100 simulated datasets analyzed with and without the effect of recycled
individuals for population size \(1000\) with \(T_2=1\) for 5 time
periods for varying survival (\(\phi=0.2,0.5,0.9\)), capture
(\(p=0.2,0.5,0.9\)), and tag retention (\(\lambda=0.2,0.5,0.9\))
probabilities. The black line indicates the true value of \(p\) used to
simulate the data for each model.

\includegraphics{Appendix_files/figure-latex/12_capture_GJSTL6-1.pdf}

\textsc{Web Figure 12:} Boxplots of capture estimates (\(\hat{p}\)) of
100 simulated datasets analyzed with and without the effect of recycled
individuals for population size \(1000\) with \(T_2=1\) for 7 time
periods for varying survival (\(\phi=0.2,0.5,0.9\)), capture
(\(p=0.2,0.5,0.9\)), and tag retention (\(\lambda=0.2,0.5,0.9\))
probabilities. The black line indicates the true value of \(p\) used to
simulate the data for each model.

\newpage

\subsection{Tag Retention Estimates}\label{tag-retention-estimates}

\includegraphics{Appendix_files/figure-latex/13_tagretention_GJSTL1-1.pdf}

\textsc{Web Figure 13:} Boxplots of tag retention estimates
(\(\hat{\lambda}\)) of 100 simulated datasets analyzed with and without
the effect of recycled individuals for population size \(1000\) with
\(T_2=1\) with 10 time periods for varying survival
(\(\phi=0.2,0.5,0.9\)), capture (\(p=0.2,0.5,0.9\)), and tag retention
(\(\lambda=0.2,0.5,0.9\)) probabilities. The black line indicates the
true value of \(\lambda\) used to simulate the data for each model.

\includegraphics{Appendix_files/figure-latex/14_tagretention_GJSTL2-1.pdf}

\textsc{Web Figure 14:} Boxplots of tag retention estimates
(\(\hat{\lambda}\)) of 100 simulated datasets analyzed with and without
the effect of recycled individuals for population size \(100000\) with
\(T_2=1\) with 10 time periods for varying survival
(\(\phi=0.2,0.5,0.9\)), capture (\(p=0.2,0.5,0.9\)), and tag retention
(\(\lambda=0.2,0.5,0.9\)) probabilities. The black line indicates the
true value of \(\lambda\) used to simulate the data for each model.

\newpage

\includegraphics{Appendix_files/figure-latex/15_tagretention_GJSTL4-1.pdf}

\textsc{Web Figure 15:} Boxplots of tag retention estimates
(\(\hat{\lambda}\)) of 100 simulated datasets analyzed with and without
the effect of recycled individuals for population size \(1000\) with
\(T_2=0.5\) with 10 time periods for varying survival
(\(\phi=0.2,0.5,0.9\)), capture (\(p=0.2,0.5,0.9\)), and tag retention
(\(\lambda=0.2,0.5,0.9\)) probabilities. The black line indicates the
true value of \(\lambda\) used to simulate the data for each model.

\includegraphics{Appendix_files/figure-latex/16_tagretention_GJSTL3-1.pdf}

\textsc{Web Figure 16:} Boxplots of tag retention estimates
(\(\hat{\lambda}\)) of 100 simulated datasets analyzed with and without
the effect of recycled individuals for population size \(100000\) with
\(T_2=0.5\) with 10 time periods for varying survival
(\(\phi=0.2,0.5,0.9\)), capture (\(p=0.2,0.5,0.9\)), and tag retention
(\(\lambda=0.2,0.5,0.9\)) probabilities. The black line indicates the
true value of \(\lambda\) used to simulate the data for each model.

\newpage

\includegraphics{Appendix_files/figure-latex/17_tagretention_GJSTL5-1.pdf}

\textsc{Web Figure 17:} Boxplots of tag retention estimates
(\(\hat{\lambda}\)) of 100 simulated datasets analyzed with and without
the effect of recycled individuals for population size \(1000\) with
\(T_2=1\) for 5 time periods for varying survival
(\(\phi=0.2,0.5,0.9\)), capture (\(p=0.2,0.5,0.9\)), and tag retention
(\(\lambda=0.2,0.5,0.9\)) probabilities. The black line indicates the
true value of \(\lambda\) used to simulate the data for each model.

\includegraphics{Appendix_files/figure-latex/18_tagretention_GJSTL6-1.pdf}

\textsc{Web Figure 18:} Boxplots of tag retention estimates
(\(\hat{\lambda}\)) of 100 simulated datasets analyzed with and without
the effect of recycled individuals for population size \(1000\) with
\(T_2=1\) for 7 time periods for varying survival
(\(\phi=0.2,0.5,0.9\)), capture (\(p=0.2,0.5,0.9\)), and tag retention
(\(\lambda=0.2,0.5,0.9\)) probabilities. The black line indicates the
true value of \(\lambda\) used to simulate the data for each model.

\newpage

\subsection{Super-Population Size
Estimates}\label{super-population-size-estimates}

\includegraphics{Appendix_files/figure-latex/19_superN_GJSTL1-1.pdf}

\textsc{Web Figure 19:} Boxplots of super-population size estimates
(\(N\)) of 100 simulated datasets analyzed with and without the effect
of recycled individuals for population size \(1000\) with \(T_2=1\) with
10 time periods for varying survival (\(\phi=0.2,0.5,0.9\)), capture
(\(p=0.2,0.5,0.9\)), and tag retention (\(\lambda=0.2,0.5,0.9\))
probabilities. The black line indicates the true value of \(N\) used to
simulate the data for each model.

\includegraphics{Appendix_files/figure-latex/20_superN_GJSTL2-1.pdf}

\textsc{Web Figure 20:} Boxplots of super-population size estimates
(\(N\)) of 100 simulated datasets analyzed with and without the effect
of recycled individuals for population size \(100000\) with \(T_2=1\)
with 10 time periods for varying survival (\(\phi=0.2,0.5,0.9\)),
capture (\(p=0.2,0.5,0.9\)), and tag retention (\(\lambda=0.2,0.5,0.9\))
probabilities. The black line indicates the true value of \(N\) used to
simulate the data for each model.

\newpage

\includegraphics{Appendix_files/figure-latex/21_superN_GJSTL4-1.pdf}

\textsc{Web Figure 21:} Boxplots of super-population size estimates
(\(N\)) of 100 simulated datasets analyzed with and without the effect
of recycled individuals for population size \(1000\) with \(T_2=0.5\)
with 10 time periods for varying survival (\(\phi=0.2,0.5,0.9\)),
capture (\(p=0.2,0.5,0.9\)), and tag retention (\(\lambda=0.2,0.5,0.9\))
probabilities. The black line indicates the true value of \(N\) used to
simulate the data for each model.

\includegraphics{Appendix_files/figure-latex/22_superN_GJSTL3-1.pdf}

\textsc{Web Figure 22:} Boxplots of super-population size estimates
(\(N\)) of 100 simulated datasets analyzed with and without the effect
of recycled individuals for population size \(100000\) with \(T_2=0.5\)
with 10 time periods for varying survival (\(\phi=0.2,0.5,0.9\)),
capture (\(p=0.2,0.5,0.9\)), and tag retention (\(\lambda=0.2,0.5,0.9\))
probabilities. The black line indicates the true value of \(N\) used to
simulate the data for each model.

\newpage

\includegraphics{Appendix_files/figure-latex/23_superN_GJSTL5-1.pdf}

\textsc{Web Figure 23:} Boxplots of super-population size estimates
(\(N\)) of 100 simulated datasets analyzed with and without the effect
of recycled individuals for population size \(1000\) with \(T_2=1\) for
5 time periods for varying survival (\(\phi=0.2,0.5,0.9\)), capture
(\(p=0.2,0.5,0.9\)), and tag retention (\(\lambda=0.2,0.5,0.9\))
probabilities. The black line indicates the true value of \(N\) used to
simulate the data for each model.

\includegraphics{Appendix_files/figure-latex/24_superN_GJSTL6-1.pdf}

\textsc{Web Figure 24:} Boxplots of super-population size estimates
(\(N\)) of 100 simulated datasets analyzed with and without the effect
of recycled individuals for population size \(1000\) with \(T_2=1\) for
7 time periods for varying survival (\(\phi=0.2,0.5,0.9\)), capture
(\(p=0.2,0.5,0.9\)), and tag retention (\(\lambda=0.2,0.5,0.9\))
probabilities. The black line indicates the true value of \(N\) used to
simulate the data for each model.

\newpage

\subsection{Abundance Estimates}\label{abundance-estimates}

\includegraphics{Appendix_files/figure-latex/25_abundance_L_GJSTL1-1.pdf}

\textsc{Web Figure 25:} Mean abundance estimates (\(N_j\)'s) for each
sample time (\(k=10\)) between analysis with and without recycled
individuals with population size \(N=1000\) with \(T_2=1\) with 10 time
periods for low tag retention (\(\lambda=0.2\)), varying survival
probabilities (\(\phi=0.2,0.5,0.9\)) and varying capture probabilities
(\(p=0.2,0.5,0.9\)).

\includegraphics{Appendix_files/figure-latex/26_abundance_L_GJSTL2-1.pdf}

\textsc{Web Figure 26:} Mean abundance estimates (\(N_j\)'s) for each
sample time (\(k=10\)) between analysis with and without recycled
individuals with population size \(N=100000\) with \(T_2=1\) with 10
time periods for low tag retention (\(\lambda=0.2\)), varying survival
probabilities (\(\phi=0.2,0.5,0.9\)) and varying capture probabilities
(\(p=0.2,0.5,0.9\)).

\newpage

\includegraphics{Appendix_files/figure-latex/27_abundance_L_GJSTL4-1.pdf}

\textsc{Web Figure 27:} Mean abundance estimates (\(N_j\)'s) for each
sample time (\(k=10\)) between analysis with and without recycled
individuals with population size \(N=1000\) with \(T_2=0.5\) with 10
time periods for low tag retention (\(\lambda=0.2\)), varying survival
probabilities (\(\phi=0.2,0.5,0.9\)) and varying capture probabilities
(\(p=0.2,0.5,0.9\)).

\includegraphics{Appendix_files/figure-latex/28_abundance_L_GJSTL3-1.pdf}

\textsc{Web Figure 28:} Mean abundance estimates (\(N_j\)'s) for each
sample time (\(k=10\)) between analysis with and without recycled
individuals with population size \(N=100000\) with \(T_2=0.5\) with 10
time periods for low tag retention (\(\lambda=0.2\)), varying survival
probabilities (\(\phi=0.2,0.5,0.9\)) and varying capture probabilities
(\(p=0.2,0.5,0.9\)).

\newpage

\includegraphics{Appendix_files/figure-latex/29_abundance_L_GJSTL5-1.pdf}

\textsc{Web Figure 29:} Mean abundance estimates (\(N_j\)'s) for each
sample time (\(k=5\)) between analysis with and without recycled
individuals with population size \(N=1000\) with \(T_2=1\) with 5 time
periods for low tag retention (\(\lambda=0.2\)), varying survival
probabilities (\(\phi=0.2,0.5,0.9\)) and varying capture probabilities
(\(p=0.2,0.5,0.9\)).

\includegraphics{Appendix_files/figure-latex/30_abundance_L_GJSTL6-1.pdf}

\textsc{Web Figure 30:} Mean abundance estimates (\(N_j\)'s) for each
sample time (\(k=7\)) between analysis with and without recycled
individuals with population size \(N=1000\) with \(T_2=1\) with 7 time
periods for low tag retention (\(\lambda=0.2\)), varying survival
probabilities (\(\phi=0.2,0.5,0.9\)) and varying capture probabilities
(\(p=0.2,0.5,0.9\)).

\newpage

\includegraphics{Appendix_files/figure-latex/31_abundance_M_GJSTL1-1.pdf}

\textsc{Web Figure 31:} Mean abundance estimates (\(N_j\)'s) for each
sample time (\(k=10\)) between analysis with and without recycled
individuals with population size \(N=1000\) with \(T_2=1\) with 10 time
periods for medium tag retention (\(\lambda=0.5\)), varying survival
probabilities (\(\phi=0.2,0.5,0.9\)) and varying capture probabilities
(\(p=0.2,0.5,0.9\)).

\includegraphics{Appendix_files/figure-latex/32_abundance_M_GJSTL2-1.pdf}

\textsc{Web Figure 32:} Mean abundance estimates (\(N_j\)'s) for each
sample time (\(k=10\)) between analysis with and without recycled
individuals with population size \(N=100000\) with \(T_2=1\) with 10
time periods for medium tag retention (\(\lambda=0.5\)), varying
survival probabilities (\(\phi=0.2,0.5,0.9\)) and varying capture
probabilities (\(p=0.2,0.5,0.9\)).

\newpage

\includegraphics{Appendix_files/figure-latex/33_abundance_M_GJSTL4-1.pdf}

\textsc{Web Figure 33:} Mean abundance estimates (\(N_j\)'s) for each
sample time (\(k=10\)) between analysis with and without recycled
individuals with population size \(N=1000\) with \(T_2=0.5\) with 10
time periods for medium tag retention (\(\lambda=0.5\)), varying
survival probabilities (\(\phi=0.2,0.5,0.9\)) and varying capture
probabilities (\(p=0.2,0.5,0.9\)).

\includegraphics{Appendix_files/figure-latex/34_abundance_M_GJSTL3-1.pdf}

\textsc{Web Figure 34:} Mean abundance estimates (\(N_j\)'s) for each
sample time (\(k=10\)) between analysis with and without recycled
individuals with population size \(N=100000\) with \(T_2=0.5\) with 10
time periods for medium tag retention (\(\lambda=0.5\)), varying
survival probabilities (\(\phi=0.2,0.5,0.9\)) and varying capture
probabilities (\(p=0.2,0.5,0.9\)).

\newpage

\includegraphics{Appendix_files/figure-latex/35_abundance_M_GJSTL5-1.pdf}

\textsc{Web Figure 35:} Mean abundance estimates (\(N_j\)'s) for each
sample time (\(k=5\)) between analysis with and without recycled
individuals with population size \(N=1000\) with \(T_2=1\) with 5 time
periods for medium tag retention (\(\lambda=0.5\)), varying survival
probabilities (\(\phi=0.2,0.5,0.9\)) and varying capture probabilities
(\(p=0.2,0.5,0.9\)).

\includegraphics{Appendix_files/figure-latex/36_abundance_M_GJSTL6-1.pdf}

\textsc{Web Figure 36:} Mean abundance estimates (\(N_j\)'s) for each
sample time (\(k=7\)) between analysis with and without recycled
individuals with population size \(N=1000\) with \(T_2=1\) with 7 time
periods for medium tag retention (\(\lambda=0.5\)), varying survival
probabilities (\(\phi=0.2,0.5,0.9\)) and varying capture probabilities
(\(p=0.2,0.5,0.9\)).

\newpage

\includegraphics{Appendix_files/figure-latex/37_abundance_H_GJSTL1-1.pdf}

\textsc{Web Figure 37:} Mean abundance estimates (\(N_j\)'s) for each
sample time (\(k=10\)) between analysis with and without recycled
individuals with population size \(N=1000\) with \(T_2=1\) with 10 time
periods for high tag retention (\(\lambda=0.9\)), varying survival
probabilities (\(\phi=0.2,0.5,0.9\)) and varying capture probabilities
(\(p=0.2,0.5,0.9\)).

\includegraphics{Appendix_files/figure-latex/38_abundance_H_GJSTL2-1.pdf}

\textsc{Web Figure 38:} Mean abundance estimates (\(N_j\)'s) for each
sample time (\(k=10\)) between analysis with and without recycled
individuals with population size \(N=100000\) with \(T_2=1\) with 10
time periods for high tag retention (\(\lambda=0.9\)), varying survival
probabilities (\(\phi=0.2,0.5,0.9\)) and varying capture probabilities
(\(p=0.2,0.5,0.9\)).

\newpage

\includegraphics{Appendix_files/figure-latex/39_abundance_H_GJSTL4-1.pdf}

\textsc{Web Figure 39:} Mean abundance estimates (\(N_j\)'s) for each
sample time (\(k=10\)) between analysis with and without recycled
individuals with population size \(N=1000\) with \(T_2=0.5\) with 10
time periods for high tag retention (\(\lambda=0.9\)), varying survival
probabilities (\(\phi=0.2,0.5,0.9\)) and varying capture probabilities
(\(p=0.2,0.5,0.9\)).

\includegraphics{Appendix_files/figure-latex/40_abundance_H_GJSTL3-1.pdf}

\textsc{Web Figure 40:} Mean abundance estimates (\(N_j\)'s) for each
sample time (\(k=10\)) between analysis with and without recycled
individuals with population size \(N=100000\) with \(T_2=0.5\) with 10
time periods for high tag retention (\(\lambda=0.9\)), varying survival
probabilities (\(\phi=0.2,0.5,0.9\)) and varying capture probabilities
(\(p=0.2,0.5,0.9\)).

\newpage

\includegraphics{Appendix_files/figure-latex/41_abundance_H_GJSTL5-1.pdf}

\textsc{Web Figure 41:} Mean abundance estimates (\(N_j\)'s) for each
sample time (\(k=5\)) between analysis with and without recycled
individuals with population size \(N=1000\) with \(T_2=1\) with 5 time
periods for high tag retention (\(\lambda=0.9\)), varying survival
probabilities (\(\phi=0.2,0.5,0.9\)) and varying capture probabilities
(\(p=0.2,0.5,0.9\)).

\includegraphics{Appendix_files/figure-latex/42_abundance_H_GJSTL6-1.pdf}

\textsc{Web Figure 42:} Mean abundance estimates (\(N_j\)'s) for each
sample time (\(k=7\)) between analysis with and without recycled
individuals with population size \(N=1000\) with \(T_2=1\) with 5 time
periods for high tag retention (\(\lambda=0.9\)), varying survival
probabilities (\(\phi=0.2,0.5,0.9\)) and varying capture probabilities
(\(p=0.2,0.5,0.9\)).

\end{document}
