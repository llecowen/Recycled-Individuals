
\documentclass[12pt]{article}

\oddsidemargin -10pt \evensidemargin -10pt \marginparwidth 50pt
\marginparsep 5pt \topmargin -0.5in \textheight 8.9in \textwidth
6.3in \hoffset=.2in

\begin{document}

\noindent Laura Cowen \newline
\noindent Department of Mathematics and Statistics \newline
\noindent University of Victoria \newline
\noindent Victoria, BC \newline
\noindent Canada V8W 3P4 \newline
\noindent lcowen@uvic.ca \newline

\bigskip

\bigskip

\noindent \normalsize October 13, 2019
\bigskip





\noindent Ecology and Evolution


\bigskip


\bigskip
\noindent Dear Editor,

\bigskip

We would like to thank you for sending us this positive review.  We would like to submit revisions to the manuscript previously entitled ``The effect of recycled individuals in the Jolly-Seber model with tag loss".  We made some grammatical changes and some changes to clarify the text. We have responded to Reviewer comments in point by point form below. 

\bigskip

Regards,

\bigskip


Laura Cowen, Emily Malcolm-White and Clive McMahon


\newpage

\noindent {\bf Response to Reviewer Comments}
\bigskip


\noindent {\bf Associate Editor Comments}

We agree with the Associate Editor's suggestions for title changes and distribution of code. The title has been updated to ``Complete tag loss in capture-recapture studies affects abundance estimates: an elephant seal case study". We intend to upload our data and R scripts on Dryad. 

\bigskip



\noindent {\bf Reviewer 1 Comments}
\begin{enumerate}
\item  {\it My one recommendation would be that the authors should add a paragraph to the discussion to discuss when the issue of bias is likely to become problematic;  the example they used was instructive,  but did not give much guidance in terms of when researchers should be using a model to account for recycled individuals}

The paragraph which outlines situations when the issues of bias are likely to be problematic has been moved to the very end of the discussion section and has been expanded upon (lines 370-379) 

\end{enumerate}

\noindent {\bf Reviewer 2 Comments}
\begin{enumerate}
\item  {\it My main critique is that the language used to describe the models that recognize previously marked individuals which have lost tags (excluded or without recycled) individuals and models that treat such individuals as new captures (included or with recycled) is inconsistent and that led to confusion at times. I had to re-read statements multiple times to make sure I properly understood what a particular model represented. I recommend the authors pick one pair of terms for these two types of models (for example, included and excluded), explain them clearly, and stick with that terminology throughout the manuscript.}

We have updated the manuscript to consistently use included and excluded recycled individuals (and have abandoned the wording with and without). Hopefully this provides more consistency and clarity to the reader.
\end{enumerate}


\bigskip

\noindent Major Comments
\begin{enumerate}
\item  {\it Under Data Accessibility, it states that this study’s data and scripts will be available on Dryad upon the paper’s acceptance. In the methods (line 107) it states that code can be obtained from the second author. I would encourage the authors to upload their data and R scripts, including the JSTL model to Dryad so that others can build upon their work. Requiring people to request code from an author adds an unnecessary barrier in my opinion.}

This was a leftover statement from our submission to MEE and we clarify as follows:
``Code from this study are included with this published article (and its supplementary information files)." (Line 115)

\item  {\it The terminology for recycled individuals included/excluded and with/without recycled individuals was not always clear and it took me a few readings to get it straight in my head. In part, I think this was because at times the terms were “included/excluded” and other times it was “with/without” recycled individuals. I recommend picking one set of terms to distinguish the two model types and sticking with it.}

See above response. We have updated the manuscript to consistently use included and excluded recycled individuals. 

\item {\it For example, on lines 220-221 the text states “This bias is corrected in the analysis without the recycled individuals considered.” I did not like this phrasing because that model does “consider” the recycled individuals, it just correctly recognizes that they are not “new” individuals but had previously been marked. I have the same issue on line 353 (“excluding these recycled individuals from the analysis can improve accuracy of the abundance estimates”) because “excluding” makes me think these individuals are just thrown out of the data, rather than properly recognizing they were previously marked.}

This has been corrected to ``By recognizing recycled individuals upon recapture, this bias can be corrected.'' (Lines 226-227).  Another line has been corrected to ``In situations where it is possible, recognizing if an individual has been
captured previously (by scarring, marking, etc) can improve accuracy of the
abundance estimates.'' (Lines 381-383) to further reflect this comment. 

\item {\it Later under “Case Study: Elephant Seals” the two models are presented as 1) ignores the effects of recycled individuals and 2) recycled individuals were recognized. I would consider revising the description of 1) because saying that recycled individuals were ignored sounded to me as though they were not double counted. Perhaps explicitly label 1) and 2) with the included/excluded terminology to connect these two scenarios to earlier descriptions.}

Those descriptions do contain the words included and excluded. The words are now italicized in an attempt to make it more prominent. 

\item {\it In the discussion, the authors state (lines 342 – 345) that future studies could simulate data for more levels of the relevant parameters (survival, capture probability, tag retention) to see how recycled individuals affect parameter estimates. I think the authors chose 3 reasonable values that pretty well cover the range of possible values for these parameters. However there really is not anything stopping the current study from simulating data with different parameter values and putting those results in the appendix. My point is that I would not expect a similar future paper to be published if it just repeated the same analysis but with slightly different values of  recapture, survival, and tag retention, so if the authors think greater insight could be gained from simulating a greater range of values (or filling in the gaps between 0.2, 0.5, and 0.9), this study might be the place to present that information. }

We agree with this comment that we have covered a reasonable range of parameter values and to go into further parameter values won't likely uncover a different story.  We have reworded the lines  to read 
``For researchers with a particular population in mind, different levels of survival, capture, or tag retention could be investigated." (Line 349)

\item {\it In the simulated data, some individuals are double-tagged and some have single tags. The status of individuals after their first capture is simulated sequentially to see if an individual 1) survives from time t to t+1, 2) loses any tags and 3) is recaptured if it has at least one tag. Presumably, if an individual is recaptured and has lost one tag, researchers could add a new second tag to replace the lost tag. If the goal of a study was to look at the effects of tag loss, one probably would not re-tag. But if the main goal is to estimate abundance and/or survival, one would likely give that individual another second tag to decrease the chances it is not recognized at time t+2 and later. Indeed, that is what I do in my own work, individuals shed PIT tags but can usually be recognized as recaptures because they are branded and brands last several years. What effect would replacing lost tags (but recognizing it is the same individual, and not treating it as new) have on parameter estimates? I expect it would reduce bias in parameter estimates because an individual would have to shed both tags between captures to be completely unrecognizable as a recapture.}

This is an interesting comment and a solid question for future work in a few directions.  The reviewer is correct that building retagging into the study design would likely reduce bias in the abundance estimates as complete tag loss would be minimized.  However, to quantify this bias (or bias reduction) one would have to develop a model that incorporates retagging. We are not currently aware of a capture-recapture model that does this and certainly, the JSTL model does not currently allow for this.  We suspect that another advantage to doing this is that there would be an increase to the precision in the tag retention parameters under the assumption of equal tag retention for the new tags and the old tags.  If tag loss was modelled to increase over time from tagging, then parameter estimates would likely be less precise. We would love to discuss this more with the reviewer and encourage him/her to contact us outside of this review process to possibly collaborate on the development of a retagging model.

We have incorporated this comment into our discussion ``Alternatively, researchers could replace lost tags on a recaptured individual thereby minimizing the occurrence of complete tag loss.  Depending on model assumptions, the JSTL model may not be appropriate for a study design involving retagging.  Future work would involve extending the JSTL model to  incorporate re-tagged individuals and assess the performance of recycled individuals within this framework." (Line 385 - 389)

\noindent Minor Comments
\begin{enumerate}
\item {\it Line 53 –- Change “it’s” to “its”} 

done.

\item {\it Lines 69 –-- 71 – This sentence might be better in the “Case Study: Elephant Seals” section because it provides details about the study of one population of elephant seals. The previous sentence would be a stronger ending to the introduction.} 

done.

\item {\it Line 107 and Line 110 – R is cited as the software used for analysis twice. I think once is sufficient.} 

done.

\item {\it Line 178 – I think the opening statement of the Simulation Results section (“The relative bias of the survival estimates are biased for some parameter combinations...”) is confusing and needs to be revised to something like “Survival estimates are biased for some parameter combinations...”} 

done.

\item {\it Lines 197 –-- 200 – This sentence does not contain much information and largely exists to tell readers to look at Table 1. I prefer statements that present a result (i.e., bias was greater for scenario X than scenario Y) and then point the reader to the table. The sentence on lines 221 – 224 is a good example of this style.}

We have removed this sentence and changed the sentence on line 193 to read: ``Results are similar for both the super-population sizes of 1000 and 100 000 for all parameter combinations of survival, capture, and tag retention probabilities (see for example Table 1).


\item \textit{Figure 4 – The legend does not show that the three line types represent different values of $\lambda$, the caption just states that three different tag retention probabilities are shown in the figure. Also, the caption for Figure 4 states that three values of $\phi$ are shown, but this figure only presents relative bias for two values of $\phi$ (0.5 and 0.9).}

We have added  the $\lambda$ to the figure legend. Values of $\phi$ in the caption have been updated to only include 0.5 and 0.9. 

\item {\it Line 284 – Correct “analyzes” to “analyses.”} 

done.

\item {\it Table 2 – I would like to see the estimates of Nsuper for both scenarios included here.}

done.

\item {\it Line 348 – Remove “unsurprisingly”} 

done.

\item {\it Line 351 – The JSTL estimator of population size is only weakly affected by recycled individuals when tag retention rates are high (e.g., Figure 2, upper left panel). Saying that the estimate of population size is not affected feels like an overstatement.} 

done.

\item {\it Line 369 – I recommend revising “the JSTL model we looked at” to “the JSTL model we used” or something similar because this study did more than just look at the model.} 

done.

\item {\it Lines 372 –-- 374 – I find this statement to be too vague. It would be better to name some of these assumptions that are violated in the real world.}

Removed this line.

\item {\it Lines 374 --– 375 – The statement “... has the potential to answer and inform researchers and managers” is missing something. Has the potential to answer what? To answer researchers and managers’ questions?}

removed ``answer and''. It now reads ``has the potential to inform researchers and managers in a meaningful way''

\item {\it Line 378 – I find “manage efficiently an ever increasing list of endangered species” to be awkward. It almost sounds like the goal is to manage the list (i.e., by adding or removing species) rather than to manage populations.}

updated to ``Having more robust estimates of vital rates is especially important if we are to effectively manage populations on an ever increasing list of endangered species.''

\item {\it Line 450 – The author list for the Schwarz et al. (2012) reference appears incomplete. The middle authors are M.A. Hindell and C.R. McMahon.}

done.
\end{enumerate}

\end{enumerate}



\bigskip

\noindent {\bf Reviewer 3 Comments}
\begin{enumerate}
\item  {\it I'm not sure current format of the abstract is what the journal is expecting. I would rather prefer a detailed abstract not in bullet form. The abstract need to be expanded highlighting main contributions.   I also think that the title need to be rephrased to represent the work better. }

The bullet form abstract is leftover from our submission to MEE - we have left the format as is, as suggested by the Associate Editor.

We have rephrased the title including the suggestions of the Associate Editor to ``Complete tag loss in capture-recapture studies affects abundance estimates: an elephant seal case study."

\item {\it It would be better to restructure the order of the paper to provide a better readership. As of now, many details are in web appendix and I find it difficult to follow. For example, complete data likelihood can be brought to the paper where the likelihood development is discussed.}

We have increased the discussion of the likelihood within the manuscript.

\item {\it Simulations settings and scenarios need to be discussed in terms of selections of parameter values: For example varying capture and survival probabilities.}

We are unsure what is meant by this. We discuss the simulation settings in lines 127 to 129.  We discuss the rational for parameter values in lines 130 to 141.  We are not sure what additional information the reviewer is asking for.

\item {\it In table 2, parameter estimates with and without recycled seem to be very close but population estimates show a considerable difference. Is there any reasoning for that?}

This observation is exactly the findings we are presenting in the simulations study and throughout the paper.  We  address this issue in the discussion  (lines 335 to 338) in the context of survival probabilities.


\item {\it In JSTL, how would you account for loss?}

Is the reviewer referring to loss on capture? If so, it is possible to include a component for loss on capture in the JSTL model. However, for this study it was not included. We have removed the mention of loss on capture within the development of the likelihood as it is not pursued here. 
\end{enumerate}





\end{document}
